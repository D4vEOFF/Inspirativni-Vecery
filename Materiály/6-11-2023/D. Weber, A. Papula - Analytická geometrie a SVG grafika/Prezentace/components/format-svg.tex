\section{Formát SVG}

\begin{frame}[t,fragile]{Formát SVG}
    \begin{itemize}
        \item SVG -- Scalable Vector Graphics.
        \item Celý obrázek složen ze základních útvarů:
        \begin{itemize}
            \item body, přímky, křivky, mnohoúhelníky.
        \end{itemize}
        \item Lze neomezeně škálovat \textbf{bez ztráty kvality}.
        \item Je možné pracovat s každým objektem zvlášť.
        \item Snadná generace pomocí programu.  
        \item Nízká paměťová náročnost.
    \end{itemize}
\end{frame}

\begin{frame}[t,fragile]{Formát SVG}
    \begin{figure}[!htb]
        \centering
        \begin{minipage}{.5\textwidth}
            \centering
            \includegraphics[width=0.9\textwidth]{image/logo.pdf}
            \caption{Vektorové logo}
            \label{fig:prob1_6_2}
        \end{minipage}%
        \begin{minipage}{.5\textwidth}
            \centering
            \includegraphics[width=0.9\textwidth]{image/logo.png}
            \caption{Rastrové logo}
            \label{fig:prob1_6_1}
        \end{minipage}
    \end{figure}
\end{frame}

\begin{frame}[t,fragile]{Formát SVG}
    \begin{figure}[!htb]
        \centering
        \begin{minipage}{.5\textwidth}
            \centering
            \includegraphics[width=0.9\textwidth]{image/logo-zoom.pdf}
            \caption{Vektorové logo}
            \label{fig:prob1_6_2}
        \end{minipage}%
        \begin{minipage}{.5\textwidth}
            \centering
            \includegraphics[width=0.9\textwidth]{image/logo-zoom.png}
            \caption{Rastrové logo}
            \label{fig:prob1_6_1}
        \end{minipage}
    \end{figure}
\end{frame}

\subsection{Syntaxe}
\begin{frame}[t,fragile]{Syntaxe}
    \begin{itemize}
        \item Jde o textový formát, přípona \texttt{.svg}.
        \item Základní struktura může vypadat takto:
    \end{itemize}
    \begin{lstlisting}[language=XML]
<?xml version="1.0" encoding="UTF-8"?>
<svg xmlns="http://www.w3.org/2000/svg" version="1.1" width="100" height="100">
    ...
</svg>
    \end{lstlisting}
\end{frame}

\begin{frame}[t,fragile]{Syntaxe}{Základní objekty}
    \begin{itemize}
        \item Obdélník:
        \begin{lstlisting}[language=XML]
<rect x="10" y="20" width="50" height="20"/>
        \end{lstlisting}
        \item Kruh:
        \begin{lstlisting}[language=XML]
<circle cx="20" cy="20" r="10"/>
        \end{lstlisting}
        \item Elipsa:
        \begin{lstlisting}[language=XML]
<ellipse cx="50" cy="50" rx="40" ry="20"/>
        \end{lstlisting}
        \item Čára:
        \begin{lstlisting}[language=XML]
<line x1="0" y1="0" x2="20" y2="30"/>
        \end{lstlisting}
    \end{itemize}
\end{frame}

\begin{frame}[t,fragile]{Syntaxe}{Základní objekty}
    \begin{itemize}
        \item Polygon:
        \begin{lstlisting}[language=XML]
<polygon points="10,10 10,20 5,20"/>
        \end{lstlisting}
        \item Lomená čára:
        \begin{lstlisting}[language=XML]
<polyline points="20,20 60,40 80,100 100,10, 10,100" style="fill:none"/>
        \end{lstlisting}
    \end{itemize}
\end{frame}

\subsection{Transformace}
\begin{frame}[t,fragile]{Transformace}{Translace}
    \begin{itemize}
        \item Všechny transformace se dělají pomocí atributu \texttt{transform}.
        \item Lze využít vestavěné příkazy nebo matici transformace.
        \item Translace pomocí hodnoty atributu \texttt{translate(tx,[ty])}.
        \item V případě matice pak hodnota atributu je \texttt{matrix(1,0,0,1,tx,ty)},
        \begin{itemize}
            \item hodnoty odpovídají prvním dvěma řádkům transformační matice, zapsané po sloupcích. 
        \end{itemize}
    \end{itemize}
    
    \begin{lstlisting}[language=XML]
<rect x="10" y="20" width="50" height="20" 
    transform="translate(10,5)"/>
<rect x="10" y="20" width="50" height="20" 
    transform="matrix(1,0,0,1,10,5)"/>
    \end{lstlisting}
\end{frame}




\begin{frame}[t,fragile]{Transformace}{Škálování}
    \begin{itemize}
        \item Škálování pomocí hodnoty atributu \texttt{scale(sx,[sy])}.
        \item V případě matice pak hodnota atributu je \texttt{matrix(sy,0,0,sy,0,0)},
        \begin{itemize}
            \item hodnoty odpovídají prvním dvěma řádkům transformační matice, zapsané po sloupcích. 
        \end{itemize}
    \end{itemize}
    
    \begin{lstlisting}[language=XML]
<rect x="0" y="10" width="10" height="10" 
    transform="scale(2)"/>
<rect x="0" y="10" width="10" height="10" 
    transform="scale(10,5)"/>

<rect x="0" y="10" width="10" height="10" 
    transform="matrix(10,0,0,5,0,0)"/>
<rect x="0" y="10" width="10" height="10" 
    transform="matrix(2,0,0,2,0,0)"/>
    \end{lstlisting}
\end{frame}

\begin{frame}[t,fragile]{Transformace}{Otočení}
    \begin{itemize}
        \item Zkosení pomocí hodnoty atributu \texttt{rotate(angle,[cx,cy])}.
        \item V případě matice pak hodnota atributu je \texttt{matrix(cos($\alpha$),sin($\alpha$),-sin($\alpha$),cos($\alpha$),0,0)},
        \begin{itemize}
            \item hodnoty odpovídají prvním dvěma řádkům transformační matice, zapsané po sloupcích. 
        \end{itemize}
    \end{itemize}
    
    \begin{lstlisting}[language=XML,basicstyle={\small\ttfamily}]
<rect x="0" y="10" width="10" height="10" 
    transform="rotate(-45)"/>
<rect x="0" y="10" width="10" height="10" 
    transform="rotate(-45,5,5)"/>

<rect x="0" y="10" width="10" height="10" 
    transform="matrix(0.7071,-0.7071,0.7071,0.7071,0,0)"/>
<rect x="0" y="10" width="10" height="10" 
    transform="matrix(0.7071,-0.7071,0.7071,0.7071,0,0)"/>
    \end{lstlisting}
\end{frame}

\begin{frame}[t,fragile]{Transformace}{Zkosení}
    \begin{itemize}
        \item Zkosení pomocí hodnoty atributu \texttt{skewX(angle)} nebo \texttt{skewY(angle)}.
        \item V případě matice je hodnota atributu pro osu $x$ \texttt{matrix(1,0,tan($\alpha$),1,0,0)}\\
              a~pro osu $y$ \texttt{matrix(1,tan($\alpha$),0,1,0,0)},
        \begin{itemize}
            \item hodnoty odpovídají prvním dvěma řádkům transformační matice, zapsané po sloupcích. 
        \end{itemize}
    \end{itemize}
    
    \begin{lstlisting}[language=XML,basicstyle={\small\ttfamily}]
<rect x="0" y="10" width="10" height="10" 
    transform="skewX(45)"/>
<rect x="0" y="10" width="10" height="10" 
    transform="skewY(45)"/>

<rect x="0" y="10" width="10" height="10" 
    transform="matrix(1,0,1,1,0,0)"/>
<rect x="0" y="10" width="10" height="10" 
    transform="matrix(1,1,0,1,0,0)"/>
    \end{lstlisting}
\end{frame}