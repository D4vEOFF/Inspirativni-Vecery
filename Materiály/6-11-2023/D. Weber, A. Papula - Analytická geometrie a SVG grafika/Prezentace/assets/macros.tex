% Enumerate
%\setlist[enumerate]{topsep=0pt,itemsep=-1ex,partopsep=1ex,parsep=1ex,label=(\arabic*)}
\setbeamertemplate{enumerate items}[circle]

% \MakeOuterQuote{"}

% Colors
\definecolor{lightblue}{HTML}{009AD4}
\definecolor{darkgreen}{HTML}{0D7103}
\definecolor{lightgreen}{HTML}{68FF00}
\definecolor{darkgreen}{HTML}{00D500}
\definecolor{darkred}{HTML}{AF0B0B}
\definecolor{lightred}{HTML}{FF5100}
\definecolor{orange}{HTML}{FFE000}
\definecolor{codeblue}{HTML}{FF0055}
\definecolor{codegreen}{rgb}{0,0.6,0}
\definecolor{codegray}{rgb}{0.5,0.5,0.5}
\definecolor{codebeige}{HTML}{D4A000}
\definecolor{codepink}{HTML}{FF0055}
\definecolor{backcolour}{rgb}{0.95,0.95,0.92}

\newcommand{\markred}[1]{\textcolor{lightred}{#1}}
\newcommand{\markgreen}[1]{\textcolor{darkgreen}{#1}}
\newcommand{\markorange}[1]{\textcolor{orange}{#1}}
\newcommand{\markblue}[1]{\textcolor{lightblue}{#1}}

% Inline images
\newcommand{\inlineimgscale}{1.1}

% X and check mark
\newcommand{\cmark}{\markgreen{\ding{51}}}
\newcommand{\xmark}{\markred{\ding{55}}}

% Math
\newcommand{\R}{\mathbb{R}}
\newcommand{\C}{\mathbb{C}}
\newcommand{\N}{\mathbb{N}}
\newcommand{\Z}{\mathbb{Z}}
\newcommand{\Q}{\mathbb{Q}}

\newcommand{\compl}[1]{#1^\complement}
\newcommand{\mapping}[3]{#1:#2 \rightarrow #3}
\newcommand{\mapto}[3]{#1:#2 \mapsto #3}
\newcommand{\set}[1]{\left\{#1\right\}}

\DeclareMathOperator{\mspan}{span}
\DeclareMathOperator{\mrank}{rank}

% TODO
\newcommand{\todo}[1]{\textcolor{red}{(\noindent TODO: #1.)}}

% Code
\lstdefinestyle{python}{
    language=Python,
    basicstyle=\small\ttfamily\color{black}, % Standardschrift
    tabsize=2, % Groesse von Tabs
    extendedchars=true, %
    breaklines=true, % Zeilen werden Umgebrochen
    %keywordstyle=\color{red}\bfseries,
    %keywordstyle=[1]\textbf, % Stil der Keywords
    % keywordstyle=[2]\textbf, %
    % keywordstyle=[3]\textbf, %
    % keywordstyle=[4]\textbf, \sqrt{\sqrt{}} %
    stringstyle=\color{codebeige}\ttfamily, % Farbe der String
    showspaces=false, % Leerzeichen anzeigen ?
    showtabs=true, % Tabs anzeigen ?
    xleftmargin=17pt,
    framexleftmargin=17pt,
    framexrightmargin=5pt,
    framexbottommargin=4pt,
    commentstyle=\color{darkgreen},
    % morecomment=[s][\color{green}]{#}{},
    %backgroundcolor=\color{grey},
    showstringspaces=false, % Leerzeichen in Strings anzeigen ?
    %morekeywords={__global__} % CUDA specific keywords
    morekeywords={False, None, True, and, as, assert, async, await, break, class, continue, def, del, elif, else, except, finally, for, from, global, if, import, in, is, lambda, nonlocal, not, or, pass, raise, return, try, while, with, yield}, % list your attributes here
    keywordstyle=\color{codepink},
    identifierstyle=\color{blue}
}
\lstdefinestyle{sharpc}{
    language={csh},
    basicstyle=\small\ttfamily\color{black}, % Standardschrift
    tabsize=2, % Groesse von Tabs
    extendedchars=true, %
    breaklines=true, % Zeilen werden Umgebrochen
    %keywordstyle=\color{red}\bfseries,
    %keywordstyle=[1]\textbf, % Stil der Keywords
    % keywordstyle=[2]\textbf, %
    % keywordstyle=[3]\textbf, %
    % keywordstyle=[4]\textbf, \sqrt{\sqrt{}} %
    stringstyle=\color{blue}\ttfamily, % Farbe der String
    showspaces=false, % Leerzeichen anzeigen ?
    showtabs=true, % Tabs anzeigen ?
    xleftmargin=17pt,
    framexleftmargin=17pt,
    framexrightmargin=5pt,
    framexbottommargin=4pt,
    commentstyle=\color{blue},
    % morecomment=[s][\color{green}]{//}{},
    %backgroundcolor=\color{grey},
    showstringspaces=false, % Leerzeichen in Strings anzeigen ?
    %morekeywords={__global__} % CUDA specific keywords
    morekeywords={abstract, event, new, struct, as, explicit, null, switch, base, extern, object, this, bool, false, operator, throw, break, finally, out, true, byte, fixed, override, try, case, float, params, typeof, catch, for, private, uint, char, foreach, protected, ulong, checked, goto, public, unchecked, class, if, readonly, unsafe, const, implicit, ref, ushort, continue, in, return, using, decimal, int, sbyte, virtual, default, interface, sealed, volatile, delegate, internal, short, void, do, is, sizeof, while, double, lock, stackalloc, else, long, static, enum, namespace, string}, % list your attributes here
    keywordstyle=\color{cyan},
    identifierstyle=\color{red}
}
\lstdefinestyle{clang}{
    basicstyle=\small\ttfamily\color{white},
    language=C,
    keywordstyle=\color{codeblue},
    commentstyle=\color{codegreen},
    numberstyle=\tiny\color{codegray},
    stringstyle=\color{codebeige},
    breakatwhitespace=false,
    breaklines=true,
    captionpos=b,
    keepspaces=true,
    numbersep=5pt,
    showspaces=false,
    showstringspaces=false,
    showtabs=false,
    morekeywords={void,int,double,float,unsigned,if,else,\#include}
    tabsize=0.5
}
\lstset{escapeinside={(*}{*)},style=clang}

\newcommand{\hlcode}[1]{\colorbox{red}{#1}}